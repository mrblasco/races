\section{Introduction}\label{introduction}

Contests are a very important source of incentives in the economy. In
the US, the government regularly sponsors open competitions in order to
solve challenging problems of public health, education, energy,
environmental issues, and so on.\footnote{On a monthly basis, the web
  portal www.challenge.gov publishes calls for new online challenges
  seeking problem solvers from all around the world.} In the private
sector, firms use contests to rapidly expand their innovative ability
(conducting open innovation initiatives) and, internally, as a means to
motivate workers (bonuses, pay increases, promotions). So, understanding
how to effectively design a contest to best achieve the desired goals is
an important economic issue.

The purpose of this study is to better understand the difference between
two quite popular choices of contest design: the race (a competition to
be first) and the tournament (a competition to be best). In particular,
we try to address the following research questions: How this choice
affect contestants? When the sponsor of a competition should pick one or
the other? What is the main trade-off? How this decision interact with
other key choices of design such as the distribution of prizes among
winners?

To address these questions we proceed in two ways. First, we generalize
the incomplete-information contest model of \citet{moldovanu2001optimal}
to allow for a comparison of both the race and the tournament within a
single framework. Then, we gather experimental data in the field from
expert competitors engaged in an online programming competition, which
we use to test some of the implications of the theory.

Economists have a long tradition in studying races, of various kinds
(patent races or arms races), and tournaments. A large body of works
have investigated several aspects of contest design, including xxx, xxx,
xxxx. However, they rarely consider the race as the result of a
deliberate choice of contest design. So we do not have many results on
when and why to use a race instead of a tournament. On the other hand,
there is a wide literature on tournaments and in particular we have many
investigations on the optimal design of a tournament. These works seem
to suggest that tournaments act as incentive mechanisms to maximize
expected total or average effort of competitors.

By this perspective, we are able to show that races cannot be justified
simply by the goal of maximizing average effort. And the reason is
intuitive. A race awards a prize to first to hit a particular target.
Those who will judge the target to hard to achieve will not join the
competition and will drop out. On the contrary, those who are able to
achieve the target at low costs will not try to exceed the target. As a
result, the race is comparable to a competition with fixed ``entry
costs'' or a fixed entry requirement, where agents will decide to either
enter and pay a fixed prize, or stay out of the competition. Then, the
possible gains in terms of expected revenues from a race are limited to
those who would enter the competition and would exert less effort that
that required to hit the target. These potential benefits can be
obtained under a tournament as well by imposing a a fixed requirement to
be eligible for prizes. So, races are not chosen to maximize expected
effort of competitors, at least, in the traditional
``auction-theoretical'' sense.

What are races for? We examine a few hypothesis and we provide some
examples. First hypothesis is that the sponsor of the race is not
primarily interested in total output but also in the time to complete a
particular task. In a tournament, this type of preferences can be
satisfied by fixing a deadline. Say time within which competitors are
asked to provide their efforts. However, assuming competitors have costs
from making less time in performing a task and there complementarities
in costs, increasing the deadline in a tournament is similar to raising
the marginal cost for everyone, which might not be an optimal solution.
In a race, by contrast, increasing the deadline will affect entry but,
conditional on entry, the time to complete the task will always be less
than the deadline. Which means that those with low costs will be mostly
affected by the deadline, whereas xxxx. Which may be a superior choice
than the tournament.

To fix ideas let consider the following example. The government wants to
solve a global public health problem such as ``antibiotics resistance.''
The overuse of antibiotics leads to the phenomenon of ``resistance''
which is a loss in the power of antibiotics to treat certain infections.
This is an increasing threat for public health. The government has the
choice of making a contest to engage people in solving this problem. The
government has preferences for time in the sense that the government
wants to minimize to have the first submission. So, the government fix a
requirement in terms of costs of the solution and award a prize to the
first to meet this requirement. Example. UK governemet goes for a race.
EU xxxx goes for a tournament with a deadline in 2016. (\ldots{}) The
answer to this optimal design question relates to the cost function of
agents with respect to ``time'' and to ``effort.'' It is hard to say
which solution is better. However, it is easier to tell whether you
should have one prize or multiple prizes.

There is also a case for efficiency. Consider a platform with many
competitions. The platform may want to engage competitors for short
period of time provide that solutions are above a certain quality level.

To test our theory we further examine experimental data on competitors
making sumibssion in an online computer programming contest. We
randomized competitors into 3 groups: 1. race 2. tournament 3.
tournament with a quality requirement we study participation, timing of
submission and final scores.

We find that, as our theory suggest, participation is higher in the
tournament and lower in the race and in the tournament with entry costs.
We further find that submission are quicker in a race, whereas are
equally distributed at the end of the competition in the the tournament
and in the tournament with quality requirement. With respect to final
scores, theory predicts as trade-off between a race and a tournament in
terms of higher scores vs faster submissions. We do find that scores are
higher in the tournament but we do not find a strong trade-off in the
sense that race had comparable good quality solutions than the
tournament.

\section{Literature}\label{literature}

This paper is related to the contest theory literature
\citet{dixit1987strategic} \citet{baye2003strategic},
\citet{parreiras2010contests}, \citet{moldovanu2001optimal},
\citet{moldovanu2006contest}, \citet{siegel2009all},
\citet{siegel2014contests}. It also relates to the literature on
innovation contests \citet{taylor1995digging}, \citet{che2003optimal}.
And the personnel economics approach to contests \citet{lazear1981rank},
\citet{green1983comparison}, \citet{mary1984economic}.

Empirically, \citet{dechenaux2014survey} provide a comprehensive summary
of the experimental literature on contests and tourments. Large body of
empirical works have focused on sports contests
\citet{szymanski2003economic}. More recently, inside firms (xxx) and
online contest (xxxx).

This paper is also related to the econometrics of auctions
\citet{paarsch1992deciding}, \citet{laffont1995econometrics},
\citet{donald1996identification} and more recently
\citet{athey2011comparing}, \citet{athey2002identification}, and
\citet{athey2007nonparametric}.
