\section{Races and tournaments}\label{races-and-tournaments}

\newcommand{\note}[1]{\textcolor{red}{[#1]}} % Notes
\newcommand\ability{a_i}
\newcommand\marginaltype{\underline a}
\newcommand\cscore{c_{y}}                   % Cost functions
\newcommand\ctime{c_{t}}
\newcommand\dord[2][X]{f_{#1_{#2:n}}}  % Distribution
\newcommand\pord[2][X]{F_{#1_{#2:n}}}
\newcommand\bidscore{y^*}                   % Equilibrium
\newcommand\bidtime{t^*}
\newcommand\invertb{b^{-1}}
\newcommand\lowercost{\underline{c}}        % Parameters
\newcommand\uppercost{\bar{c}}
\newcommand\deadline{d}
\newcommand\tweight{\tau}
\newcommand\target{\bar y}
\newcommand\entrylimit{\bar\theta}
\newcommand{\Expect}{{\bf E}}      % Expectations
\newcommand\realsp{\mathbb{R}^{+}}

\subsection{The basic theoretical
model}\label{the-basic-theoretical-model}

Consider a contest with \(k\) available prizes of value
\(V_1 > V_2 > ... > V_k\). Each agent (\(i=1, ..., n+1\)) moves
simultaneously to maximize the chances of winning a prize. To be
elegible for a prize, each agent has to complete a task within a
deadline \(d>0\). Outcomes are then evaluated and ranked along two
dimensions: the output quality \(y_i\) and the time to completion
\(t_i\) both being nonnegative real numbers.

In a tournament, the agent having achieved the highest output quality
within the deadline gets the first prize, the agent having achieved the
second highest output quality gets the second prize, and so on. In a
race, by contrast, the first agent to achieve an output quality of at
least \(\target\) within the deadline wins the first prize, the second
to achieve the same target gets the second prize, and so on.

Since agents move simultaneously, they do not know the performance of
others when deciding their efforts. On the other hand, it is assumed
that they know the number of competitors as well as their cost functions
to complete the task up to a factor \(a_i\) being the agent's private
ability in performing the task. Each agent knows his ability but does
not know the ability of the others. However, it is common knowledge that
abilities are drawn at random from a common distribution \(F_A\) that is
assumed everywhere differentiable on the support
\(V\subseteq [0, \infty)\).

It is further assumed that costs are multiplicative

\begin{equation}
  C(y_i, t_i, a_i) = \cscore (y) \cdot \ctime (t)  \cdot a_i^{-1}
\end{equation}

with \(\cscore(0)\geq 0\), \(\cscore^\prime>0\), \(\ctime(d)\geq 0\),
and \(\ctime^\prime<0\).

Each agent is risk neutral and faces the following decision problem

\begin{equation}
  \begin{array}{ll}
    \mbox{maximize} & \sum_{j=1}^k \Pr(\text{ranked $j$'th}) V_j  - C(y_i, t_i, a_i).
  \end{array}
\end{equation}

\subsubsection{Equilibrium in a
tournament}\label{equilibrium-in-a-tournament}

We provide here the symmetric equilibrium with one prize
\note{todo: two prizes} and \(n>2\) agents. In appendix XXX, we provide
a general formula for \(k>2\) prizes.

Let \(y_{1:n} < y_{2:n} < ... < y_{n:n}\) denote the order statistics of
the \(y_j\)'s for every \(j\neq i\) and let \(\pord[Y]{r}(\cdot)\) and
\(\dord[Y]{r}(\cdot)\) denote the corresponding distribution and density
for the \(r\)'th order statistic.

In a tournament, the unique symmetric equilibrium of the model gives,
for every \(i=1, ..., n\), the optimal time to completion \(t^*(a_i)\)
equal to the deadline \(d\) and the optimal output quality \(y^*(a_i)\)
as

\begin{equation}
  \label{eq: optimal bid tournament}
  y^*(a_i) =  V_1 \int_{a_i}^\infty \dord[Y]{n} (z) dz
\end{equation}

if \(\ability \geq \marginaltype\) \citep[see][]{moldovanu2001optimal},
and equal to zero otherwise.

An important property of \eqref{eq: optimal bid tournament} is that
\(y^*(a_i)\) has its upper bound in \note{value} and lower bound in
\note{value}. Also, equilibrium output quality is monotonic increasing
in the agent's ability \citep[see][]{moldovanu2001optimal}. Thus, for
every \(i=1, ..., n+1\), the equilibrium expected reward depends only on
the rank of his ability relative to the others. Using \(\pord[A]{r}\) to
denote the distribution of the \(r\)'th order statistic of abilities
gives

\begin{equation}
  \label{eq: expected payoffs tournament}
  \pord[A]{n}(a_i) V_1  - C(y_i^*, d, a_i).
\end{equation}

Hence, by setting \eqref{eq: expected payoffs tournament} to zero and
solving for the ability, gives the marginal ability \(\marginaltype\) as

\begin{equation}
  \marginaltype = h(n, V, F_A, C, d). 
\end{equation}

\subsubsection{Equilibrium behavior in a
race}\label{equilibrium-behavior-in-a-race}

\ldots{}

\subsubsection{The expected revenues for the sponsor of the
contest}\label{the-expected-revenues-for-the-sponsor-of-the-contest}

The sponsor of the contest chooses the rules of the competition
including prize structure \(\{V_j\}_{j=1}^k\), deadline \(d\), target
quality \(q\), and competition format (race or tournament). The sponsor
maximizes an objective function that is the sum of total quality
\(Y=\sum_{i=1}^{n+1} Y_i\), time spent \(T=\sum_{i=1}^{n+1} T_i\) and
prizes paid \(V=\sum_{j=1}^k p_{j} V_j\) (with \(p_j=1\) if the prize is
awarded and \(p_j=0\) otherwise). Hence, the problem faced by the
sponsor is

\begin{equation}
  \begin{array}{ll}
    \mbox{maximize} & \int {Y}   -  \tweight \Expect{T} - \Expect{V}
  \end{array}
\end{equation}

with the intensity of preferences towards time weighted by
\(c_t\geq 0\).

\subsection{Structural econometric
model}\label{structural-econometric-model}
