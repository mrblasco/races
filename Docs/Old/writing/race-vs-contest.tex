\documentclass[serif]{amsart}

% ------------------------------------------------------------%
\title{Races vs. Tournaments?}
\date{First Version:~April 10, 2014\\
    Last updated:~\today}
\author{Andrea Blasco}
\thanks{Institute Quantitative Social Sciences and Nasa Tournament Laboratory}
% ------------------------------------------------------------%
\begin{document} 
\maketitle 

\section{The Model}
Consider $N$ agents competing for a prize. Agent's $i$ cost for submitting a solution is $\theta_i$ 


\section{Assumptions}
Consider a set $N =\{ 1,2,..., n\}$ of risk-neutral agents.  Agents ability is a random variable $a\in[0,1]$ with a beta-family probability mass function  $f_{[0,1]}(a;\alpha,\beta)$, that is iid across agents.
Consider that the quality of a solution is $q \in [0,\infty)$ and the time to produce a  solution of a given quality is represented by a continuous real-valued function
\[
	t(a,q) = a^{\gamma} \cdot q^{\delta}  
\]
with $\gamma<0$ so that the partial derivative of time with respect to ability $t^{\prime}(a)<0$ if $q\neq 0$ and $\delta>0$ so that the partial derivative of time with respect to quality $t^{\prime}(q)>0$. 
\footnote{Here the idea is that agents commit resources at the beginning of the race in order to deliver a solution of a given quality by a certain time. It's clear that if any of the opponents wins the race, this means that the others will have some free time ex-post. Our assumption throught the paper is that this extra-time is a cost that cannot  cannot be reinvested. 
In a more general model, we can assume that ...   }
Agents' cost of time of development is represented by strictly increasing function $c(t(a,q)) =  C \cdot t(a,q)^{\kappa}$, and with no loss of generality we set $\kappa=1$.



% ------------------------------------------------------------  %
 %% RACE 
% ------------------------------------------------------------  %
\subsection{Race}
The first agent to develop a solution that hits $q\geq \bar q$ is awarded a prize $v>0$.
Under these rules, an agent $i$'s ex-post utility is:
\[
U_i = \left\{  
	\begin{array}{cl}
		v - c(t(a_i,q_i)) & if q_i \geq \bar q ~ and ~ t(a_i,q_i) < t(a_j,q_j)~\forall j \in N_i =\{k \in N/{i}:q_k\geq\bar q\} \\
		-  c(t(a_i,q_i))  & otherwise 	
	\end{array}
\right.
\] 
At an interim stage, the agent learns his type $a_i$ and has to decide how much quality to develop (or equivalently how much time to develop a solution of a given quality).
The maximization problem is (forget about ties): 
\[
	\max_{q_i\in A} ~ \mathbf{1}_{(q_i\geq \bar q)}\cdot \Pr( t(a_i,q_i) < \min\{t(a_j,q_j)\}_{j \in N_k} | a_i ) \cdot v - c(t(a_i,q_i))
\]
where $\mathbf{1}()$ is an indicator function.
Note, fixed other agents' strategies, it is easy to show that the above problem reduces to a binary decision on $q\in\{0,\bar q\}$, as all other actions are (strictly) dominated. Indeed, the probability of winning is non-increasing in quality for any level $q_i>\bar q$ and recall that costs are strictly increasing in $q_i$. So this can be rewritten:
\[
	\max_{q_i\in \{0,\bar q\}} ~ \mathbf{1}_{(q_i = \bar q)}\cdot F_{A}(a_i > a_j)^{n-1} \cdot v - c(t(a_i,q_i))
\]

Furthermore, because the above expresion is monotone in $a_i$, we can simply look at a the marginal agent to pin down a threshold level of ability $\hat a$ above which all agents are going to bid $q^* = \bar q$ and below which they will bid $q^* = 0$. That is,
\[
	v \cdot \left( \frac{B(\hat a,\alpha,\beta)}{B(\alpha,\gamma)}\right)^{n-1}  = C \cdot \hat a^{\gamma} \cdot \bar q^{\delta}   
\]
which simplifies to the following remarkably simple expression if we assume that abilities are distributed uniformly on the unit interval:
\[
	 \hat a  = \left( \frac {C \cdot \bar q ^\delta}{v}\right)^{1/(n-1-\gamma)}
\]
notice that when $\delta / (n-1-\gamma) > 1 $, the threshold grows exponentially in $\bar q$, whereas when $\delta / (n-1-\gamma) <1 $ it grows logarithmically (or linear for the equal). This has a clear interpretation ... 
 

For a principal, the problem is the following:
\[
	w_{race} =  \Pr(\max \{q_i\}_{i \in N } = \bar q)\cdot (\bar q - v)
\]
which for the uniform unit interval case becomes:
\[
	w_{race} =  \left[ 1 -  \left(\frac{C \cdot \bar q ^\delta}{v}\right)^{n/(n-1-\gamma)}  \right] \cdot (\bar q - v)
\]

% ------------------------------------------------------------  %
 %% ALL PAY AUCTION
% ------------------------------------------------------------  %
\section{Equilibrium in the All-pay tournament}
The problem here is the following:
\[
	\underset{q}{\arg\max} \qquad v \cdot \Pr\left( q_i  > \max \{q_j\}_{j\neq i}) \right) - c(t(a_i,q_i))
\]
and by following M-S (for $n>2$ and uniform unit interval) we have:
\[
	q_{allpay}^* =  \left(\frac{n-1}{n-1+\gamma} v  \cdot a^{n-1+\gamma}\right)^{1/\delta}
\]
 
\section{Which one is betterd?}
We are in the shoes of a ``principal'' i.e., sponsor/crowdsourcing platform.
First, suppose the principal does not care about time per se, but only as a mean to provide good incentives to participants or to save costs. 
Second, the principal wants to max the expected \emph{max} quality. Third, the principal needs to optimally set the incentives. For example, in a race the problem of a principal is:
\[
	\max_{\rho \leq 1} ~ \hat{w}^{races} = w/V =  \left\{1 - [1-F(\bar x(\rho))]^{N} \right\} \cdot (\rho  - 1)
\]
On the other hand, in a tournament the only parameter to change is $V$. 
First, given the bidding function, we want to know the expected min cost in the race:
\[
	F(\min(x,y)<k) = F(x<k)*F(y<k) = F(k)^2
\]
Hence, for N=2,
\[
	\int_{t_min}^1 F(k)^2 d k = \int_{t_m}^1  \frac{k-t_m}{1-t_m} dk   	=  \frac{1 - t_m}2 
\]
And the expected max value  is 
\[
	\hat w^{allpay} = w/V =  \frac{log(2/(1-t_m))}{1-t_m}  - 1
\]


\section{Description of the data}

 
\end{document}