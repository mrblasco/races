\documentclass[12pt, titlepage, draft]{article} % 12pt, titlepage, draft
%%%%%%%%%%%%%%%%%%%%%%%%%%%%%%%%%%%%%%%%%%%%%%%%%%%%%%%
\usepackage[utf8]{inputenc}                           %
\usepackage[english]{babel}                           %
\usepackage{amsmath,amssymb,amsfonts}                 % 
\usepackage{xcolor,tikz,graphicx,subfig}              % Figures
\usepackage{rotating,multirow,array,dcolumn,booktabs} % Tables  
\usepackage{natbib}                                   % Bibliography
\bibliographystyle{aer}                               % Bib-style
%%%%%%%%%%%%%%%%%%%%%%%%%%%%%%%%%%%%%%%%%%%%%%%%%%%%%%%

% New commands
%% Math environments
\newtheorem{proposition}{Proposition}
\newtheorem{define}{Definition}


% Thresholds, limits, bounds, etc.
\newcommand\deadline{t_{0}}
\newcommand\target{y_{0}}

% Competitions
\newcommand\race{\text{race}}
\newcommand\tournament{\text{tournament}}
\newcommand\reserve{\text{reserve}}

% Cost functions
\newcommand\ctime{c_{\tau}}
\newcommand\cscore{c_{y}}
\newcommand\cability{c_{a}}
%\newcommand\costs{\cability(a_i)\cscore(y_i)\ctime(t_i)}
\newcommand\costs{C(a_i, y_i, t_i)}

\newcommand\ability{a_i}
\newcommand\quality{q_i}
\newcommand\speed{t_i}

\newcommand\marginaltype{\hat{a}}
\newcommand\mtype{\hat{a}}
\newcommand\lotype{\underline{a}}
\newcommand\hitype{\bar{a}}



% Derivatives
\newcommand\dystar{\frac{\partial y^*(x,\target)}{\partial\target}dF_{N:N}(x)}


% MANAGEMENT SCIENCE %%%%%%%%%%%%%%%%%%%%%%%%%%%%%%%%%%
%\usepackage[letterpaper, margin=1in]{geometry}  % 1'' margins 
\usepackage{setspace}                           
  \onehalfspacing %\doublespacing or \singlespacing
\usepackage{endfloat}
%\makeatletter
%\renewcommand{\@maketitle}{
%\newpage
%\null
%\vskip 2em%
%\begin{center}%
%{\LARGE \@title \par}%
%\vskip 2em%
%{\@date \par}%
%\end{center}%
%\par}
%\makeatother
%%%%%%%%%%%%%%%%%%%%%%%%%%%%%%%%%%%%%%%%%%%%%%%%%%%%%%%%%


\title{%
Races or Tournaments? [Draft: Please do not distribute]\thanks{Blasco: Harvard Institute for Quantitative Social Science, Harvard University, 1737 Cambridge Street, Cambridge, MA 02138 (email: ablasco@fas.harvard.edu)}
}
\author{%
Andrea Blasco \and Kevin J. Boudreau \and Karim R. Lakhani \and Michael Menietti
}
\date{%
This version: \today
}

\begin{document}
\maketitle
\tableofcontents

% Abstract
\begin{abstract}

\noindent A wide range of economic and social situations are decided by either a
race or a tournament. In such contests, agents choose whether and how
much to exert some costly effort to increase the probability of being
awarded a prize under uncertainty about the other agents types or
actions. In theory, whenever the sponsor of the competition prefers
competitors' performance over the time to complete a particular task,
the expected outcomes of a tournament setup should be either equal or
greater than those of a race. Yet, a race might be more efficient from
an economic point of view as it may prevent unnecessary costs due to an
excess of participation. We examine this trade-off empirically. We
report the results of a field experiment conducted on a leading
crowdsourcing platform where we compare the outcomes (efforts, quality,
and diversity of outputs) of three alternative competitive situations
motivated by theory: the race, the tournament, and the tournament with a
quality requirement.

\smallskip\noindent 
JEL Classification: XX; XX; XX;

\smallskip\noindent 
Keywords: xx; xx; xx;
\end{abstract}


% Content
\clearpage
\input{1_intro.tex}
\input{2_methods.tex}
\input{3_results.tex}
\input{4_discussion.tex}
  
% Bibliography
\newpage
\singlespacing
\bibliography{refs}
\end{document}
