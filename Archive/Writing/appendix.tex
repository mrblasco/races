\documentclass[11pt]{article} % 12pt, titlepage, draft
%%%%%%%%%%%%%%%%%%%%%%%%%%%%%%%%%%%%%%%%%%%%%%%%%%%%%%%
\usepackage[utf8]{inputenc}                           %
\usepackage[english]{babel}                           %
\usepackage{amsmath,amssymb,amsfonts}                 % 
\usepackage{xcolor,tikz,graphicx,subfig}              % %\graphicspath{{}}
\usepackage{rotating,multirow,array,dcolumn,booktabs} % booktabs  
\usepackage{natbib}                                   % Bibliography
\bibliographystyle{aer}                               % Bib-style
%%%%%%%%%%%%%%%%%%%%%%%%%%%%%%%%%%%%%%%%%%%%%%%%%%%%%%%

% New commands
% Math environments
\newtheorem{proposition}{Proposition}
\newtheorem{define}{Definition}


% Thresholds, limits, bounds, etc.
\newcommand\deadline{t_{0}}
\newcommand\target{y_{0}}

% Competitions
\newcommand\race{\text{race}}
\newcommand\tournament{\text{tournament}}
\newcommand\reserve{\text{reserve}}

% Cost functions
\newcommand\ctime{c_{\tau}}
\newcommand\cscore{c_{y}}
\newcommand\cability{c_{a}}
%\newcommand\costs{\cability(a_i)\cscore(y_i)\ctime(t_i)}
\newcommand\costs{C(a_i, y_i, t_i)}

\newcommand\ability{a_i}
\newcommand\quality{q_i}
\newcommand\speed{t_i}

\newcommand\marginaltype{\hat{a}}
\newcommand\mtype{\hat{a}}
\newcommand\lotype{\underline{a}}
\newcommand\hitype{\bar{a}}



% Derivatives
\newcommand\dystar{\frac{\partial y^*(x,\target)}{\partial\target}dF_{N:N}(x)}



% MANAGEMENT SCIENCE %%%%%%%%%%%%%%%%%%%%%%%%%%%%%%%%%%
\usepackage[letterpaper, margin=1in]{geometry}  % 1'' margins 
\usepackage{setspace}                           
\onehalfspacing %\doublespacing or \singlespacing
%\usepackage{endfloat}
\makeatletter
\renewcommand{\@maketitle}{
\newpage
\null
\vskip 2em%
\begin{center}%
{\LARGE \@title \par}%
\vskip 2em%
{\@date \par}%
\end{center}%
\par}
\makeatother
%%%%%%%%%%%%%%%%%%%%%%%%%%%%%%%%%%%%%%%%%%%%%%%%%%%%%%%%%


\title{%
Appendix -- Contributing to Public Goods Inside Organizations: Field Experimental Evidence
}
\author{%
by Andrea Blasco, Olivia S. Jung, Karim R. Lakhani and Michael Menietti\thanks{Blasco: Harvard Institute for Quantitative Social Science, Harvard University, 1737 Cambridge Street, Cambridge, MA 02138 (email: ablasco@fas.harvard.edu). Jung: Harvard Business School, Soldiers Field, Boston, MA 02163 (email: ojung@hbs.edu), Lakhani: Harvard Business School, Soldiers Field, Boston, MA 02163, and National Bureau of Economic Research (email: k@hbs.edu). Menietti: Harvard Institute for Quantitative Social Science, Harvard University, 1737 Cambridge Street, Cambridge, MA 02138 (email: mmenietti@fas.harvard.edu). We gratefully acknowledge the financial support of the MacArthur Foundation (Opening Governance Network), NASA Tournament Lab, and the Harvard Business School Division of Faculty Research and Development.  This project would not have been possible without the support of Eric Isselbacher, Julia Jackson, Maulik Majmudar and Perry Band from the Massachusetts General Hospital's Healthcare Transformation Lab.}
}
\date{%
This version: \today
}

\begin{document}
\maketitle

% Abstract
\begin{abstract}
In section \ref{extended-model-with-heterogenous-costs} of this online appendix, we present a formal proof of the result of sorting in the extended model with heterogenous costs, as discussed in the main paper. In section \ref{experimental-design-graphics} we include copy of the solicitation sent and the graphics used for the website. 
\end{abstract}



% Content
%\clearpage
%\input{1_intro.tex}
%\input{2_methods.tex}
%\input{3_results.tex}
%\input{4_discussion.tex}
  
%Bibliography
%\newpage
%\singlespacing
%\bibliography{refs}

% Appendix
\appendix
\section{Online Appendix}

\subsection{Extended model with heterogenous costs}
\label{extended-model-with-heterogenous-costs}

In this section, consider the case of two types of individuals $j=1,2$ forming two groups of equal size $n_1=n_2=n$. Individuals can decide to contribute with a single proposal or not.  

When an agent of type $j$ decides to contribute, the expected utility is as follows. 
\begin{equation}
 u_1^j = \gamma \hat Y + \delta_j + \sum_{k_j=1}^n \sum_{k_l=0}^n \Pr(Y=k_j+k_l) \frac{R}{k_j+k_l} - c_j.
\end{equation}
The utility of not contributing is as follows. 
\begin{equation}
 u_0^j = \gamma (\hat Y - 1).
\end{equation}
Equating these two conditions for all individuals gives the following mixed-strategy equilibrium condition:
\begin{equation}
\sum_{k_j=1}^n \sum_{k_l=0}^n \Pr(Y=k_j+k_l) \frac{R}{k_j+k_l} = c_j -\delta_j + \gamma
\end{equation}
for all $j=1,2$. To examine differences in equilibrium probabilities $p_1^*$ and $p_2^*$, we use the ratio between the above equilibrium condition for individuals of type $j=1$ and the same expression for agents of type $j=2$. This gives: 
\begin{equation}
\frac{\sum_{k_1=1}^n \sum_{k_2=0}^n \Pr(Y=k_1+k_2) \frac{R}{k_1+k_2}}{\sum_{k_1=0}^n \sum_{k_2=1}^n \Pr(Y=k_1+k_2) \frac{R}{k_1+k_2}} = \frac{c_1 -\delta_1 + \gamma}{c_2 -\delta_2 + \gamma}.
\end{equation}
The left hand side can be rearranged as follows.
\begin{equation}
\frac{\Pr(k_2=0) \sum_{k_1=1}^n\Pr(Y=k_1)\frac{R}{k_1} +  \sigma R}{\Pr(k_1=0) \sum_{k_2=1}^n\Pr(Y=k_2)\frac{R}{k_2} +  \sigma R} 
\end{equation}
where $\sigma = \sum_{k_1=1}^n \sum_{k_2=1}^n \Pr(Y=k_1+k_2) \frac{1}{k_1+k_2}$.
Using $1-p_2=\Pr(k_2=0)$ and  $1-p_1=\Pr(k_1=0)$ together with the density of the binomial distribution, we obtain the following simpler expression.
\begin{equation}
\frac{(1-p_2) \frac{(1- (1-p_1)^n)}{n p_1} R +  \sigma R }{(1-p_1) \frac{(1- (1-p_2)^n)}{n p_2} R +  \sigma R}  .
\end{equation}
If $c_1 - \delta_1 > c_2 - \delta_2$, then the above expression in equilibrium needs to be larger than one. This inequality can be expressed as follows:
\begin{equation}
\frac{p_2 (1-p_2)}{(1- (1-p_2)^n)}  > \frac{p_1 (1-p_1)}{(1- (1-p_1)^n)}.
\end{equation}
Hence, the inequality is satisfied only if $p_2$ is greater than $p_1$. This proves the last statement reported in the Section with predictions in this paper. 

\subsection{Experimental design graphics}\label{experimental-design-graphics}

Figure \ref{fig: solicitation} shows the copy of one solicitation sent. Figure \ref{fig: priming} shows the graphics used in the challenge website.

\begin{figure} \centering
\caption{Copy of the solicitation email sent at the beginning of the submission phase}
\label{fig: solicitation}
\includegraphics[width=\textwidth]{figures/solicitationEmail.pdf}
\end{figure}

\begin{figure} \centering
\caption{Graphics displayed on the contest's website matching the treatment}
\label{fig: priming}
\begin{tabular}{cc}
\includegraphics[width=2in, height=1.5in]{figures/priming/funding.png} & 
\includegraphics[width=2in, height=1.5in]{figures/priming/money.png} \\
FUND & PRIZE \\
\includegraphics[width=2in, height=1.5in]{figures/priming/patientcare.png} & 
\includegraphics[width=2in, height=1.5in]{figures/priming/workplace.png} \\
PCARE & WPLACE
\end{tabular}
\end{figure}

\end{document}
